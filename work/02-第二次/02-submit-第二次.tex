\documentclass[lang = zh]{iwork}

\title{第二次}
\course{Advanced Numerical Analysis}
\institute{School of Software, Tsinghua University}
\email{li-q24@mails.tsinghua.edu.cn}
\id{2024312371}
\class{}

\def\bookname{《数值分析基础 (第二版) (关治, 陆金甫)》}

\begin{document}

\maketitle

\begin{prob}[\bookname P226. 1.]
  设 $f(x) = e^x, \ x \in [0, 2]$
  \begin{enumerate}
    \item $x_0 = 0.0,\,x_1 = 0.5$, 构造一次 Lagrange 插值多项式 $L_1$, 并计算 $L_1(0.25)$.
    \item $x_0 = 0.5,\,x_1 = 1.0$, 构造一次 Lagrange 插值多项式 $L_1$, 并计算 $L_1(0.75)$.
    \item $x_0 = 0.0,\,x_1 = 1.0,\,x_2 = 2.0$, 构造二次 Lagrange 插值多项式 $L_2$ 并计算 $L_2(0.25), L2(0.75)$.
  \end{enumerate}
\end{prob}
\begin{sol} ~
  \begin{enumerate}
    \item ~ \vspace{-\baselineskip}
          \begin{align*}
            L_{10}    & = \frac{x - x_1}{x_0 - x_1}                                                         \\
            L_{11}    & = \frac{x - x_0}{x_1 - x_0}                                                         \\
            L_1       & = f(x_0) L_{10}(x) + f(x_1) L_{11}(x)                                               \\
                      & = e^{0.0} \cdot \frac{x - 0.5}{0.0 - 0.5} + e^{0.5} \cdot \frac{x - 0.0}{0.5 - 0.0} \\
                      & = 1 + (2 e^{0.5} - 2) x                                                             \\
            L_1(0.25) & = 1 + (2 e^{0.5} - 2) \cdot 0.25 \approx \num{1.324}
          \end{align*}
    \item ~ \vspace{-\baselineskip}
          \begin{align*}
            L_1       & = e^{0.5} \cdot \frac{x - 1.0}{0.5 - 1.0} + e^{1.0} \cdot \frac{x - 0.5}{1.0 - 0.5} \\
                      & = 2 e^{0.5} - e + (2 e - 2 e^{0.5}) x                                               \\
            L_1(0.75) & = 2 e^{0.5} - e + (2 e - 2 e^{0.5}) \cdot 0.75 \approx \num{2.184}
          \end{align*}
    \item ~ \vspace{-\baselineskip}
          \begin{align*}
            L_{20} & = \frac{(x - x_1)(x - x_2)}{(x_0 - x_1)(x_0 - x_2)} \\
            L_{21} & = \frac{(x - x_0)(x - x_2)}{(x_1 - x_0)(x_1 - x_2)} \\
            L_{22} & = \frac{(x - x_0)(x - x_1)}{(x_2 - x_0)(x_2 - x_1)}
          \end{align*}
          \begin{equation*}
            \begin{split}
              L_2 & = f(x_0) L_{20}(x) + f(x_1) L_{21}(x) + f(x_2) L_{22}(x)                                                                                      \\
                  & = e^0 \cdot \frac{(x - 1)(x - 2)}{(0 - 1)(0 - 2)} + e^{0.5} \cdot \frac{x(x - 2)}{(1 - 0)(1 - 2)} + e^1 \cdot \frac{x(x - 1)}{(2 - 0)(2 - 1)} \\
                  & = \frac{1}{2} (x - 1) (x - 2) - e x (x - 2) + \frac{e^2}{2} x (x - 1)                                                                         \\
            \end{split}
          \end{equation*}
          \begin{equation*}
            L_2(0.25) \approx \num{1.153},\;
            L_2(0.75) \approx \num{2.012}
          \end{equation*}
  \end{enumerate}
\end{sol}

\begin{prob}[\bookname P226. 4.]
  设 $l_0, l_1, \cdots, l_n$ 是以节点 $x_0, x_1, \cdots, x_n$ 的 $n$ 次 Lagrange 插值基函数, 试证明
  \begin{equation*}
    \sum_{k = 0}^n l_k(0) x_k^j =
    \begin{cases}
      0,                         & j = 1, 2, \cdots, n, \\
      (-1)^n x_0 x_1 \cdots x_n, & j = n + 1.
    \end{cases}
  \end{equation*}
\end{prob}
\begin{proof}
  令 $f(x) = x^j$, 则 $n$ 次插值多项式为 $L_n(x) = \sum_{k = 0}^n l_k(x) x_k^j$.
  当 $j = 1, 2, \cdots, n$ 时, 由插值多项式的性质可知, $\sum_{k = 0}^n l_k(0) x_k^j = L_n(0) = f(0) = 0$.
  当 $j = n + 1$ 时, 对于 $\bar{x} = 0$, 存在 $\xi \in I[x_0, x_1, \cdots, x_n, 0]$, 使得
  \begin{equation*}
    \sum_{k = 0}^n l_k(0) x_k^j
    = f(0) - L_n(0)
    = \frac{f^{n + 1}(\xi)}{(n + 1)!} (0 - x_0) \cdots (0 - x_n)
    = (-1)^n x_0 x_1 \cdots x_n
  \end{equation*}
\end{proof}

\begin{prob}[\bookname P226. 5.]
  设 $f(x) = x^5 + 4 x^3 + 1$, 试求均差 $f[0,1,2],\, f[0,1,2,3,4,5],\, f[0,1,2,3,4,5,6]$.
\end{prob}
\begin{sol} ~ \vspace{-\baselineskip}
  \begin{table}
    \centering
    \begin{tabular}{c|ccccccc}
                & $k = 0$ & 1    & 2    & 3   & 4  & 5 & 6 \\
      \hline
      $x_0 = 0$ & 1                                        \\
      $x_1 = 1$ & 6       & 5                              \\
      $x_2 = 2$ & 65      & 59   & 27                      \\
      $x_3 = 3$ & 352     & 287  & 114  & 29               \\
      $x_4 = 4$ & 1281    & 929  & 321  & 69  & 10         \\
      $x_5 = 5$ & 3626    & 2345 & 708  & 129 & 15 & 1     \\
      $x_6 = 6$ & 8641    & 5015 & 1335 & 209 & 20 & 1 & 0
    \end{tabular}
  \end{table}
  \vspace{-2\baselineskip}
  \begin{gather*}
    f[0,1,2] = 27      \\
    f[0,1,2,3,4,5] = 1 \\
    f[0,1,2,3,4,5,6] = 0
  \end{gather*}
\end{sol}

\begin{prob}[\bookname P227. 12.]
  求次数不高于三次的多项式 $P$ 使其满足 $p(0) = p'(0) = 0,\, p(1) = 1,\, p(2) = 1$, 并写出其 Newton 形式的余项.
\end{prob}
\begin{sol} ~ \vspace{-\baselineskip}
  \begin{table}
    \centering
    \begin{tabular}{ccccc}
      0          & 1 & 2              & 3              \\
      \midrule
      $p(0) = 0$ &                                     \\
      $p(0) = 0$ & 0                                   \\
      $p(1) = 1$ & 1 & 1                               \\
      $p(2) = 1$ & 0 & $-\frac{1}{2}$ & $-\frac{3}{4}$ \\
      \bottomrule
    \end{tabular}
  \end{table}
  \vspace{-2\baselineskip}
  \begin{gather*}
    P(x) = x^2 - \frac{3}{4} x^2 (x - 1) \\
    R_3(x) = f[0,0,1,2,x] x^2 (x - 1) (x - 2)
  \end{gather*}
\end{sol}

\begin{prob}[\bookname P227. 13.]
  求次数不超过 4 次的多项式 $P$, 使其满足
  \begin{equation*}
    p(1) = p(3) = 0,\;
    p(2) = 1,\;
    p'(1) = 0,\;
    p''(1) = 8;
  \end{equation*}
  并写出其 Newton 形式的余项.
\end{prob}
\begin{sol} ~ \vspace{-\baselineskip}
  \begin{table}
    \centering
    \begin{tabular}{ccccc}
      0          & 1           & 2            & 3  & 4 \\
      \midrule
      $p(1) = 0$ &                                     \\
      $p(1) = 0$ & $p'(1) = 0$ &                       \\
      $p(1) = 0$ & $p'(1) = 0$ & $p''(1) = 8$          \\
      $p(2) = 1$ & 1           & 1            & -7     \\
      $p(3) = 0$ & -1          & -1           & -1 & 3 \\
      \bottomrule
    \end{tabular}
  \end{table}
  \vspace{-2\baselineskip}
  \begin{gather*}
    P_4(x) = 8 (x - 1)^2 - 7 (x - 1)^3 + 3 (x - 1)^3 (x - 2) \\
    R_4(x) = f[1,1,1,2,3,x](x - 1)^3 (x - 2) (x - 3)
  \end{gather*}
\end{sol}

\begin{prob}[\bookname P227. 14.]
  设 $f(x) = \frac{1}{a - x}$, 证明
  \begin{enumerate}
    \item \label{prob:14.1} $\displaystyle f[x_0,x_1,\cdots,x_n] = \prod_{j = 0}^n \pqty{\frac{1}{a - x_j}}$.
    \item \label{prob:14.2} $\displaystyle \frac{1}{a - x} = \frac{1}{a - x_0} + \frac{1}{(a - x_0) (a - x_1)} (x - x_0) + \cdots + \frac{1}{(a - x_0) \cdots (a - x_n)} (x - x_0) \cdots (x - x_{n - 1}) + \frac{1}{(a - x_0) \cdots (a - x_n) (a - x)} (x - x_0) \cdots (x - x_n)$.
  \end{enumerate}
  (提示: \cref{prob:14.1} 用归纳法; \cref{prob:14.2} Newton 插值多项式)
\end{prob}
\begin{proof} ~
  \begin{enumerate}
    \item 当 $n = 0$ 时,
          \begin{equation*}
            f[x_0] = f(x_0) = \frac{1}{a - x_0}
          \end{equation*}
          假设当 $n = k$ 时, $f[x_0,x_1,\cdots,x_k] = \prod_{j = 0}^k \pqty{\frac{1}{a - x_j}}$ 成立, 则当 $n = k + 1$ 时,
          \begin{equation*}
            \begin{split}
              f[x_0,x_1,\cdots,x_{k + 1}]
               & = \frac{f[x_1,\cdots,x_{k + 1}] - f[x_0,\cdots,x_k]}{x_{k + 1} - x_0}                                                        \\
               & = \frac{1}{x_{k + 1} - x_0} \pqty{\prod_{j = 1}^{k + 1} \pqty{\frac{1}{a - x_k}} - \prod_{j = 0}^k \pqty{\frac{1}{a - x_k}}} \\
               & = \frac{1}{x_{k + 1} - x_0} \pqty{\frac{1}{a - x_{k + 1}} - \frac{1}{a - x_0}} \prod_{j = 1}^k \pqty{\frac{1}{a - x_k}}      \\
               & = \prod_{j = 0}^{k + 1} \pqty{\frac{1}{a - x_k}}
            \end{split}
          \end{equation*}
          由数学归纳法可知, $\forall n \in \mathbb{N}$, $f[x_0,x_1,\cdots,x_n] = \prod_{j = 0}^n \pqty{\frac{1}{a - x_j}}$.
    \item Newton 插值多项式为
          \begin{equation*}
            \begin{split}
              P_n(x)
               & = f[x_0] + f[x_0,x_1] (x - x_0) + \cdots + f[x_0,x_1,\cdots,x_n] (x - x_0) \cdots (x - x_{n - 1})                                              \\
               & = \frac{1}{a - x_0} + \frac{1}{(a - x_0) (a - x_1)} (x - x_0) + \cdots + \frac{1}{(a - x_0) \cdots (a - x_n)} (x - x_0) \cdots (x - x_{n - 1})
            \end{split}
          \end{equation*}
          余项为
          \begin{equation*}
            R_n(x) = f[x_0,x_1,\cdots,x_n,x] (x - x_0) \cdots (x - x_n)
          \end{equation*}
          由 \cref{prob:14.1} 可知, $f[x_0,x_1,\cdots,x_n,x] = \prod_{j = 0}^n \pqty{\frac{1}{a - x_j}} \cdot \frac{1}{a - x} = \frac{1}{(a - x_0) \cdots (a - x_n) (a - x)}$.
          故
          \begin{equation*}
            R_n(x) = \frac{1}{(a - x_0) \cdots (a - x_n) (a - x)} (x - x_0) \cdots (x - x_n)
          \end{equation*}
          由 $f(x) = \frac{1}{a - x}$ 可知, $\frac{1}{a - x} = P_n(x) + R_n(x)$.
  \end{enumerate}
\end{proof}

\begin{prob}[\bookname P227. 15.]
  设 $f(x) = a_0 + a_1 x + \cdots + a_{n - 1} x^{n - 1} + a_n x^n$ 有 $n$ 个不同的实根 $x_1, x_2, \cdots, x_n$, 试证明
  \begin{equation*}
    \sum_{j = 1}^n \frac{x_j^k}{f'(x_j)} =
    \begin{cases}
      0,        & 0 \leqslant k \leqslant n - 2, \\
      a_n^{-1}, & k = n - 1.
    \end{cases}
  \end{equation*}
\end{prob}
\begin{proof}
  不妨设 $f(x) = a_n (x - x_1) (x - x_2) \cdots (x - x_n)$.
  令 $P(x) = x^k$, $w_n(x) = (x - x_1) (x - x_2) \cdots (x - x_n)$.
  则存在 $\xi \in I(x_0, x_1, \cdots, x_n)$, 使得
  \begin{equation*}
    P[x_1,x_2,\cdots,x_n]
    = \frac{P^{(n)}(\xi)}{n!}
    = P[x_1,x_2,\cdots,x_n]
    = \sum_{j = 1}^n \frac{P(x_j)}{w_n'(x_j)}
    = \frac{1}{a_n} \sum_{j = 1}^n \frac{x_j^k}{f'(x_j)}
  \end{equation*}
  当 $0 \leqslant k \leqslant n - 2$ 时, $P^{(n)}(\xi) = 0$, $\sum_{j = 1}^n \frac{x_j^k}{f'(x_j)} = 0$. \\
  当 $k = n - 1$ 时, $P^{(n)}(\xi) = n!$, $\sum_{j = 1}^n \frac{x_j^k}{f'(x_j)} = a_n^{-1}$.
\end{proof}

\end{document}
