\documentclass[lang = zh]{iwork}
\usepackage{anyfontsize} % https://tex.stackexchange.com/a/328322
\usepackage{bm}
\usepackage{todonotes}

\DeclareDocumentCommand{\booktitle}{}{《数值分析基础 (第二版) (关治, 陆金甫)》}
\setlist[enumerate]{label = (\arabic*)}

\course{Advanced Numerical Analysis (60420024-1)}
\title{ODE solver}
\ymd{2024}{12}{27}

\begin{document}

\maketitle

\begin{prob}[\booktitle P404 第九章 \ 常微分方程初值问题的数值解法 4.]
  用梯形方法解初值问题 $y' = -y$, $y(0) = 1$, 试证明:
  \begin{enumerate}
    \item 取 $y_0 = y(0) = 1$, 有 $y_n = \left(\frac{2 - h}{2 + h}\right)^n$.
    \item 当 $h \to 0$, $x_n = n h$ 不变时, $y_n$ 收敛于初值问题的准确解 $e^{-x_n}$.
  \end{enumerate}
\end{prob}
\begin{proof} ~
  % 提示:
  % (2)
  % \begin{equation*}
  %   \lim_{h \to 0}\left(\frac{2 - h}{2 + h}\right)^n
  %   = \lim_{h \to 0}\left(\frac{1 - \frac{h}{2}}{1 + \frac{h}{2}}\right)^n
  %   = \lim_{n \to \infty} \frac{\left(1 - \frac{x_n}{2 n}\right)^n}{\left(1 + \frac{x_n}{2 n}\right)^n}
  %   = e^{-x_n}.
  % \end{equation*}
  \begin{enumerate}
    \item 梯形方法的递推公式为:
          \begin{equation*}
            y_{n+1} = y_n + \frac{h}{2} \left( f(x_n, y_n) + f(x_{n+1}, y_{n+1}) \right)
          \end{equation*}
          对于 $y' = -y$, 有 $f(x, y) = -y$, 因此递推公式变为:
          \begin{equation*}
            y_{n+1} = y_n + \frac{h}{2} \left( -y_n - y_{n+1} \right)
          \end{equation*}
          整理得到:
          \begin{equation*}
            y_{n+1} \left(1 + \frac{h}{2}\right) = y_n \left(1 - \frac{h}{2}\right)
          \end{equation*}
          于是:
          \begin{equation*}
            y_{n+1} = \frac{1 - \frac{h}{2}}{1 + \frac{h}{2}} y_n
          \end{equation*}
          这是一个等比数列递推关系, 初始条件为 $y_0 = 1$, 因此:
          \begin{equation*}
            y_n
            = \left(\frac{1 - \frac{h}{2}}{1 + \frac{h}{2}}\right)^n
            = \left(\frac{2 - h}{2 + h}\right)^n
          \end{equation*}
    \item 当 $h \to 0$ 时, $x_n = n h$ 保持不变.
          我们需要证明:
          \begin{equation*}
            \lim_{h \to 0} \left(\frac{2 - h}{2 + h}\right)^n = e^{-x_n}
          \end{equation*}
          令 $h = \frac{x_n}{n}$, 则:
          \begin{equation*}
            \lim_{h \to 0} \left(\frac{2 - h}{2 + h}\right)^n = \lim_{n \to \infty} \left(\frac{2 - \frac{x_n}{n}}{2 + \frac{x_n}{n}}\right)^n
          \end{equation*}
          将分子和分母同时除以 2:
          \begin{equation*}
            \lim_{n \to \infty} \left(\frac{1 - \frac{x_n}{2n}}{1 + \frac{x_n}{2n}}\right)^n
          \end{equation*}
          利用极限公式 $\lim_{n \to \infty} \left(1 + \frac{a}{n}\right)^n = e^a$, 我们有:
          \begin{equation*}
            \lim_{n \to \infty} \left(1 - \frac{x_n}{2n}\right)^n = e^{-\frac{x_n}{2}}, \quad
            \lim_{n \to \infty} \left(1 + \frac{x_n}{2n}\right)^n = e^{\frac{x_n}{2}}
          \end{equation*}
          因此:
          \begin{equation*}
            \lim_{n \to \infty} \left(\frac{1 - \frac{x_n}{2n}}{1 + \frac{x_n}{2n}}\right)^n
            = \frac{e^{-\frac{x_n}{2}}}{e^{\frac{x_n}{2}}}
            = e^{-x_n}
          \end{equation*}
          即:
          \begin{equation*}
            \lim_{h \to 0} \left(\frac{2 - h}{2 + h}\right)^n = e^{-x_n}
          \end{equation*}
          这表明当 $h \to 0$ 时, 梯形方法的数值解 $y_n$ 收敛于初值问题的准确解 $e^{-x_n}$.
  \end{enumerate}
\end{proof}

\begin{prob}[\booktitle P404 第九章 \ 常微分方程初值问题的数值解法 5.]
  试求出单步法
  \begin{equation*}
    \begin{cases}
      y_{n + 1} = y_n + h f(x_{n + 1}, y_n + h f(x_n, y_n)) \\
      y_0 = y(x_0)
    \end{cases}
  \end{equation*}
  的局部截断误差主项及绝对稳定性区间.
\end{prob}
\begin{sol} ~
  % 答案: $T_{n + 1} = - \frac{h^2}{2} y''(x_n)$; 绝对稳定性区间为 $(-1, 0)$
  \paragraph*{局部截断误差主项}
  假设精确解为 $y(x)$, 则有:
  \begin{equation*}
    y(x_{n+1})
    = y(x_n + h)
    = y(x_n) + h y'(x_n) + \frac{h^2}{2} y''(x_n) + \frac{h^3}{6} y'''(x_n) + \cdots
  \end{equation*}
  根据微分方程 $y'(x_n) = f(x_n, y(x_n))$, 所以:
  \begin{equation*}
    y(x_{n+1}) = y(x_n) + h f(x_n, y(x_n)) + \frac{h^2}{2} f'(x_n, y(x_n)) + \cdots
  \end{equation*}
  而数值方法给出:
  \begin{equation*}
    y_{n+1} = y_n + h f(x_{n+1}, y_n + h f(x_n, y_n))
  \end{equation*}
  展开 $f(x_{n+1}, y_n + h f(x_n, y_n))$:
  \begin{equation*}
    f(x_{n+1}, y_n + h f(x_n, y_n))
    = f(x_n + h, y_n + h f(x_n, y_n))
    = f(x_n, y_n) + h \frac{\partial f}{\partial x}(x_n, y_n) + h f(x_n, y_n) \frac{\partial f}{\partial y}(x_n, y_n) + \cdots
  \end{equation*}
  因此:
  \begin{equation*}
    y_{n+1} = y_n + h \left[ f(x_n, y_n) + h \frac{\partial f}{\partial x}(x_n, y_n) + h f(x_n, y_n) \frac{\partial f}{\partial y}(x_n, y_n) + \cdots \right]
  \end{equation*}
  比较精确解和数值解的展开式, 得到局部截断误差的主项:
  \begin{equation*}
    t_{n+1}
    = y(x_{n+1}) - y_{n+1}
    = \left( y_n + h f + \frac{h^2}{2} (f_x + f f_y) \right) - \left( y_n + h f + h^2 (f_x + f f_y) \right) + \cdots = - \frac{h^2}{2} y''(x_n)
  \end{equation*}
  \paragraph*{绝对稳定性区间}
  考虑线性测试方程 $y' = \lambda y$, 其中 \(\lambda\) 是复数.
  令 $f(x, y) = \lambda y$, 则:
  \begin{equation*}
    y_{n+1}
    = y_n + h f(x_{n+1}, y_n + h f(x_n, y_n))
    = y_n + h \lambda (y_n + h \lambda y_n) = y_n (1 + h \lambda + h^2 \lambda^2)
  \end{equation*}
  绝对稳定性要求 $|1 + h \lambda + h^2 \lambda^2| < 1$.
  假设 \(\lambda = \mu\) 是实数, 则:
  \begin{equation*}
    1 + h \mu + h^2 \mu^2 < 1 \quad \text{且} \quad 1 + h \mu + h^2 \mu^2 > -1
  \end{equation*}
  首先, $1 + h \mu + h^2 \mu^2 < 1$, 即 $h \mu (1 + h \mu) < 0$, 所以 \(-1 < h \mu < 0\).
  其次, $1 + h \mu + h^2 \mu^2 > -1$ 总是成立.
  因此, 绝对稳定性区间是 $(-1, 0)$.
\end{sol}

\begin{prob}[\booktitle P404 第九章 \ 常微分方程初值问题的数值解法 6.]
  应用中点公式及 Heun 方法
  \begin{equation*}
    y_{n + 1} = y_n + \frac{1}{4} h f(x_n, y_n) + \frac{3}{4} h f(x_n + \frac{2}{3} h, y_n + \frac{2}{3} h f(x_n, y_n))
  \end{equation*}
  计算初值问题
  \begin{equation*}
    y' = y - x^2 + 1, \quad
    x \in [0, 1], \quad
    y(0) = 0.5.
  \end{equation*}
  取 $h = 0.2$, 列出数值解及相应的误差 (问题的解析解为 $y(x) = (x + 1)^2 - \frac{1}{2} e^x$.).
\end{prob}
\begin{sol} ~
  % 答案:
  % \begin{table}[htbp]
  %   \sisetup{table-alignment-mode = none}
  %   \centering
  %   \begin{tabular}{SSSSSS} \null
  %     {$x_n$} & {$y(x_n)$} & {中点公式} & {误差}    & {Heun 方法} & {误差}    \\
  %     0       & 0.5000000  & 0.5000000  & 0         & 0.5000000   & 0         \\
  %     0.2     & 0.8292986  & 0.8280000  & 0.0012986 & 0.8273333   & 0.0019653 \\
  %     0.4     & 1.2140877  & 1.2113600  & 0.0027277 & 1.2098800   & 0.0042077 \\
  %     0.6     & 1.6489406  & 1.6446592  & 0.0042814 & 1.6421869   & 0.0067537 \\
  %     0.8     & 2.1272295  & 2.1212842  & 0.0059453 & 2.1176014   & 0.0096281 \\
  %     1.0     & 2.6408591  & 2.6331668  & 0.0076923 & 2.6280070   & 0.0128521
  %   \end{tabular}
  % \end{table}
  \begin{table}[htbp]
    \sisetup{table-alignment-mode = none}
    \centering
    \begin{tabular}{SSSSSS} \null
      {$x_n$} & {$y(x_n)$} & {中点公式} & {误差}    & {Heun 方法} & {误差}    \\
      0       & 0.5000000  & 0.5000000  & 0         & 0.5000000   & 0         \\
      0.2     & 0.8292986  & 0.8280000  & 0.0012986 & 0.8273333   & 0.0019653 \\
      0.4     & 1.2140877  & 1.2113600  & 0.0027277 & 1.2098800   & 0.0042077 \\
      0.6     & 1.6489406  & 1.6446592  & 0.0042814 & 1.6421869   & 0.0067537 \\
      0.8     & 2.1272295  & 2.1212842  & 0.0059453 & 2.1176014   & 0.0096281 \\
      1.0     & 2.6408591  & 2.6331668  & 0.0076923 & 2.6280070   & 0.0128521
    \end{tabular}
  \end{table}
\end{sol}

\begin{prob}[\booktitle P404 第九章 \ 常微分方程初值问题的数值解法 8.]
  试求出中点公式
  \begin{equation*}
    y_{n + 1} = y_n + h \trigbraces{f}(x_n + \frac{h}{2}, y_n + \frac{1}{2} h f(x_n, y_n))
  \end{equation*}
  的局部截断误差主项.
\end{prob}
\begin{sol}
  % 答案:
  % \begin{equation*}
  %   T_{n + 1} = h^3 \left[\frac{1}{6} y'' \pdv{f}{y} + \frac{1}{24} \left(\pdv[2]{f}{x} + 2 f \pdv{f}{x}{y} + f^2 \pdv[2]{f}{y}\right)\right]_{(x_n, y(x_n))}
  % \end{equation*}
  首先, 我们考虑中点公式:
  \begin{equation*} y_{n+1} = y_n + h f\left(x_n + \frac{h}{2}, y_n + \frac{1}{2} h f(x_n, y_n)\right) \end{equation*}
  我们需要求出其局部截断误差的主项.
  为此, 我们将精确解 $y(x_{n+1})$ 展开成泰勒级数, 并与中点公式的表达式进行比较.
  精确解的泰勒展开为:
  \begin{equation*}
    y(x_n + h) = y(x_n) + h y'(x_n) + \frac{h^2}{2} y''(x_n) + \frac{h^3}{6} y'''(x_n) + \cdots
  \end{equation*}
  其中,
  \begin{gather*}
    y'(x_n) = f(x_n, y_n)  \\
    y''(x_n) = f_x + f_y f \\
    y'''(x_n) = f_{xx} + 2 f_{xy} f + f_x f_y + f_{yy} f^2 + f_y^2 f
  \end{gather*}
  中点公式的右边展开为:
  \begin{equation*}
    y_{n+1} = y_n + h \left[ f(x_n, y_n) + \frac{h}{2} f_x + \frac{h}{2} f f_y + \frac{h^2}{8} f_{xx} + \frac{h^2}{4} f_{xy} f + \frac{h^2}{8} f^2 f_{yy} + \cdots \right]
  \end{equation*}
  比较精确解和中点公式的展开式, 得到局部截断误差的主项:
  \begin{equation*}
    \begin{split}
      T_{n+1}
       & = y(x_n + h) - y_{n+1}                                                                                                                                                               \\
       & = h^3 \left( \frac{1}{6} (f_{xx} + 2 f_{xy} f + f_x f_y + f_{yy} f^2 + f_y^2 f) - \left( \frac{1}{8} f_{xx} + \frac{1}{4} f_{xy} f + \frac{1}{8} f^2 f_{yy} \right) \right) + \cdots
    \end{split}
  \end{equation*}
  化简后得到:
  \begin{equation*} T_{n+1} = h^3 \left[ \frac{1}{6} y'' \pdv{f}{y} + \frac{1}{24} \left( \pdv[2]{f}{x} + 2 f \pdv{f}{x y} + f^2 \pdv[2]{f}{y} \right) \right]_{(x_n, y(x_n))} \end{equation*}
\end{sol}

\begin{prob}[\booktitle P404 第九章 \ 常微分方程初值问题的数值解法 10.]
  试求出隐式中点方法
  \begin{equation*}
    y_{n + 1} = y_n + h \trigbraces{f}(x_n + \frac{1}{2} h, \frac{1}{2} (y_n + y_{n - 1}))
  \end{equation*}
  的绝对稳定性区间 (推广 \S\,3.5 节方法).
\end{prob}
\begin{sol}
  % 答案: 用于试验方程 $y' = \lambda y$, 可得绝对稳定性区间为 $(-\infty, 0)$.
  首先, 我们考虑隐式中点方法:
  \begin{equation*}
    y_{n+1} = y_n + h f\left(x_n + \frac{1}{2} h, \frac{1}{2} (y_n + y_{n+1})\right)
  \end{equation*}
  对于测试方程 $y' = \lambda y$, 我们有 $f(x, y) = \lambda y$, 因此方法变为:
  \begin{equation*}
    y_{n+1} = y_n + \frac{h \lambda}{2} (y_n + y_{n+1})
  \end{equation*}
  整理得到:
  \begin{equation*}
    y_{n+1} \left(1 - \frac{h \lambda}{2}\right) = y_n \left(1 + \frac{h \lambda}{2}\right)
  \end{equation*}
  解出 $y_{n+1}$:
  \begin{equation*}
    y_{n+1} = y_n \cdot \frac{1 + \frac{h \lambda}{2}}{1 - \frac{h \lambda}{2}}
  \end{equation*}
  记 $R = \frac{1 + \frac{h \lambda}{2}}{1 - \frac{h \lambda}{2}}$, 为了保证绝对稳定性, 要求 $|R| \leqslant 1$.
  设 $z = h \lambda$, 则 $R = \frac{1 + \frac{z}{2}}{1 - \frac{z}{2}}$, 要求:
  \begin{equation*}
    \left| \frac{1 + \frac{z}{2}}{1 - \frac{z}{2}} \right| \leqslant 1
  \end{equation*}
  展开模的平方:
  \begin{equation*}
    \left|1 + \frac{z}{2}\right|^2 \leqslant \left|1 - \frac{z}{2}\right|^2
  \end{equation*}
  设 $z = \mu + i \nu$, 展开并简化得到:
  \begin{equation*}
    1 + \mu + \frac{\mu^2}{4} + \frac{\nu^2}{4} \leqslant 1 - \mu + \frac{\mu^2}{4} + \frac{\nu^2}{4}
  \end{equation*}
  消去相同项后得到:
  \begin{equation*}
    \mu \leqslant -\mu
  \end{equation*}
  即:
  \begin{equation*}
    \mu \leqslant 0
  \end{equation*}
  因此, 只要 $\Re(z) \leqslant 0$, 即 $\Re(h \lambda) \leqslant 0$, 方法是绝对稳定的.
  综上所述, 隐式中点方法的绝对稳定性区间为:
  \begin{equation*}
    (-\infty, 0)
  \end{equation*}
\end{sol}

\end{document}
