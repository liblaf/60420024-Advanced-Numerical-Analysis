\documentclass[lang = zh]{iwork}
\usepackage{anyfontsize} % https://tex.stackexchange.com/a/328322
\usepackage{bm}
\usepackage{todonotes}

\DeclareDocumentCommand{\booktitle}{}{《数值分析基础 (第二版) (关治, 陆金甫)》}
\setlist[enumerate]{label = (\arabic*)}

\course{Advanced Numerical Analysis (60420024-1)}
\title{最后一次作业}

\begin{document}

\maketitle

\begin{prob}[\booktitle P133 第四章 16.]
  设 $a > 0$, 证明迭代公式
  \begin{equation*}
    x_{k + 1} = \frac{x_k (x_k^2 + 3 a)}{3 x_k^2 + a}
  \end{equation*}
  产生的序列 $\{x_k\}$ 三阶收敛到 $\sqrt{a}$.

  % 提示: $\varphi(\sqrt{a}) = \sqrt{a}$, $\varphi'(\sqrt{a}) = \varphi''(\sqrt{a}) = 0$, $\varphi'''(\sqrt{a}) \neq 0$.
\end{prob}
\begin{sol}
  首先, 我们验证迭代公式的不动点:
  \begin{equation*}
    \varphi(\sqrt{a})
    = \frac{\sqrt{a} (\sqrt{a}^2 + 3 a)}{3 \sqrt{a}^2 + a}
    = \sqrt{a}
  \end{equation*}
  接下来, 求导并验证导数在不动点处的值:
  \begin{equation*}
    \varphi'(x)
    = \frac{(3 x^2 + 3 a) (3 x^2 + a) - (x^3 + 3 a x) (6 x)}{(3 x^2 + a)^2}
    = \frac{3 (a - x^2)^2}{(a + 3 x^2)^2}
  \end{equation*}
  代入 $x = \sqrt{a}$:
  \begin{equation*}
    \varphi'(\sqrt{a})
    = \frac{3 (a - a)^2}{(a + 3 a)^2}
    = 0
  \end{equation*}
  再求二阶导数并验证:
  \begin{equation*}
    \varphi''(x) = \frac{48 a x (x^2 - a)}{(a + 3 x^2)^3}
  \end{equation*}
  代入 $x = \sqrt{a}$:
  \begin{equation*}
    \varphi''(\sqrt{a}) = \frac{48 a \sqrt{a} (a - a)}{(a + 3 a)^3} = 0
  \end{equation*}
  最后, 求三阶导数并验证:
  \begin{equation*}
    \varphi'''(x)
    = \frac{48a(3x^2 - a)(3x^2 + a)^3 - 48a(x^3 - a x) \cdot 18x(3x^2 + a)^2}{(3x^2 + a)^6}
    = - \frac{48 a (a^2 - 18 a x + 9 x^4)}{(a + 3 x^2)^4}
  \end{equation*}
  代入 $x = \sqrt{a}$:
  \begin{equation*}
    \varphi'''(\sqrt{a})
    = - \frac{48 a (a^2 - 18 a \sqrt{a} + 9 a^2)}{(a + 3 a)^4}
    \neq 0
  \end{equation*}
  根据迭代收敛的阶数理论, 如果 $\varphi(l) = l$, $\varphi'(l) = 0$, $\varphi''(l) = 0$, $\varphi'''(l) \neq 0$, 那么迭代是三阶收敛的.
  因此, 序列 $\{x_k\}$ 三阶收敛到 $\sqrt{a}$.
\end{sol}

\begin{prob}[\booktitle P133 第四章 18.]
  设 $\bm{A} \in \mathbb{R}^{n \times n}$, $\det{\bm{A}} \neq 0$. $\mathbb{R}^{n \times n}$ 中矩阵序列 $\{\bm{X}_k\}$ 满足:
  \begin{equation*}
    \bm{X}_{k + 1} = \bm{X}_k + \bm{X}_k (\bm{I} - \bm{A} \bm{X}_k) \qc
    k = 0, 1, \cdots.
  \end{equation*}
  试证明:
  \begin{enumerate}
    \item 令 $\bm{E}_k = \bm{I} - \bm{A} \bm{X}_k$, 则 $\bm{E}_{k + 1} = \bm{E}_k^2$.
    \item 若 $\norm{\bm{I} - \bm{A} \bm{X}_0} < 1$, 则 $\{\bm{X}_k\}$ 收敛到 $\bm{A}^{-1}$.
    \item 将向量序列收敛阶的概念推广到矩阵序列, 说明上述求 $\bm{A}^{-1}$ 的迭代法至少二阶收敛.
  \end{enumerate}

  % 提示:
  % (2) $\norm{\bm{X}_k - \bm{A}^{-1}} = \norm{\bm{A}^{-1} (\bm{A} \bm{X}_k - \bm{I})} \leqslant \norm{\bm{A}^{-1}} \norm{\bm{E}_k} \leqslant \cdots \leqslant \norm{\bm{A}^{-1}} \norm{\bm{E}_0}^{2^k}$
  % (3) $\bm{X}_k - \bm{A}^{-1} = - \bm{A}^{-1} \bm{E}_{k - 1}^2$
\end{prob}
\begin{sol} ~
  \begin{enumerate}
    \item \label{sol:2.1} 定义 $\bm{E}_k = \bm{I} - \bm{A} \bm{X}_k$, 则有:
          \begin{equation*}
            \bm{E}_{k+1}
            = \bm{I} - \bm{A} \bm{X}_{k+1}
            = \bm{I} - \bm{A} (\bm{X}_k + \bm{X}_k (\bm{I} - \bm{A} \bm{X}_k))
            = \bm{I} - \bm{A} \bm{X}_k - \bm{A} \bm{X}_k (\bm{I} - \bm{A} \bm{X}_k)
          \end{equation*}
          化简得:
          \begin{equation*}
            \bm{E}_{k+1}
            = \bm{E}_k - \bm{A} \bm{X}_k \bm{E}_k
            = \bm{E}_k - (\bm{I} - \bm{E}_k) \bm{E}_k
            = \bm{E}_k^2
          \end{equation*}
          因此, $\bm{E}_{k+1} = \bm{E}_k^2$.
    \item \label{sol:2.2} 若 $\norm{\bm{I} - \bm{A} \bm{X}_0} < 1$, 即 $\norm{\bm{E}_0} < 1$, 则根据 \ref{sol:2.1} 的结果, 有:
          \begin{equation*}
            \bm{E}_{k} = \bm{E}_0^{2^k}
          \end{equation*}
          因此:
          \begin{equation*}
            \norm{\bm{E}_k} \leqslant \norm{\bm{E}_0}^{2^k}
          \end{equation*}
          考虑 $\bm{X}_k - \bm{A}^{-1}$:
          \begin{equation*}
            \bm{X}_k - \bm{A}^{-1}
            = \bm{A}^{-1} (\bm{A} \bm{X}_k - \bm{I})
            = \bm{A}^{-1} \bm{E}_k
          \end{equation*}
          所以:
          \begin{equation*}
            \norm{\bm{X}_k - \bm{A}^{-1}}
            \leqslant \norm{\bm{A}^{-1}} \norm{\bm{E}_k}
            \leqslant \norm{\bm{A}^{-1}} \norm{\bm{E}_0}^{2^k}
          \end{equation*}
          当 $k \to \infty$ 时, $\norm{\bm{E}_0}^{2^k} \to 0$, 因为 $\norm{\bm{E}_0} < 1$.
          因此, $\{\bm{X}_k\}$ 收敛到 $\bm{A}^{-1}$.
    \item 将向量序列收敛阶的概念推广到矩阵序列, 若存在常数 $C$ 使得:
          \begin{equation*}
            \norm{\bm{X}_{k+1} - \bm{A}^{-1}} \leqslant C \norm{\bm{X}_k - \bm{A}^{-1}}^2
          \end{equation*}
          则称该迭代法至少二阶收敛.
          根据 \ref{sol:2.2}, 有:
          \begin{equation*}
            \bm{X}_k - \bm{A}^{-1} = -\bm{A}^{-1} \bm{E}_{k-1}^2
          \end{equation*}
          因此:
          \begin{equation*}
            \norm{\bm{X}_k - \bm{A}^{-1}}
            \leqslant \norm{\bm{A}^{-1}} \norm{\bm{E}_{k-1}}^2
            \leqslant \norm{\bm{A}^{-1}} (\norm{\bm{E}_0}^{2^{k-1}})^2
            = \norm{\bm{A}^{-1}} \norm{\bm{E}_0}^{2^k}
          \end{equation*}
          这表明每一步的误差大约是前一步误差的平方, 故该迭代法至少二阶收敛.
  \end{enumerate}
\end{sol}

\begin{prob}[\booktitle P133 第四章 13.]
  \begin{enumerate}
    \item 为求 $f(x) = 0$ 的根, 用迭代函数 $\varphi(x) = x + f(x)$ 的迭代法不一定收敛.
          对此用 Steffensen 加速方法, 试写出迭代公式.
    \item 设 $f$ 有连续的二阶导数, $f(x^*) = 0$, $f'(x^*) \neq 0$, 研究迭代法
          \begin{equation*}
            x_{k + 1} = x_k - \frac{[f(x_k)]^2}{f(x_k + f(x_k)) - f(x_k)} \qc
            k = 0, 1, \cdots
          \end{equation*}
          的收敛性和收敛阶.
  \end{enumerate}
\end{prob}
\begin{sol} ~
  % 提示:
  % (1) 迭代公式即 (2).
  % (2) 至少二阶收敛.
  \begin{enumerate}
    \item 给定迭代函数 $\varphi(x) = x + f(x)$, Steffensen 加速公式为:
          \begin{equation*}
            \trigbraces{f}(x)
            x_{k+1} = x_k - \frac{[f(x_k)]^2}{f(x_k + f(x_k)) - f(x_k)}
          \end{equation*}
    \item 由 Steffensen 迭代公式, 有
          \begin{equation*}
            \varphi(x) = x - \frac{[f(x)]^2}{f(x + f(x)) - f(x)}
          \end{equation*}
          对 $f(x + f(x))$ 进行泰勒展开, 得
          \begin{equation*}
            f(x + f(x)) = f(x) + f'(x) f(x) + \frac{1}{2} f''(\xi) [f(x)]^2
          \end{equation*}
          其中, $\xi$ 是 $x$ 和 $x + f(x)$ 之间的一个值.
          于是, 有
          \begin{equation*}
            g(x)
            = \frac{f(x + f(x)) - f(x)}{f(x)}
            = \frac{f'(x) f(x) + \frac{1}{2} f''(\xi) [f(x)]^2}{f(x)}
            = f'(x) + \frac{1}{2} f''(\xi) f(x)
          \end{equation*}
          进而,
          \begin{equation*}
            \varphi(x)
            = x - \frac{[f(x)]^2}{f(x + f(x)) - f(x)}
            = x - \frac{f(x)}{f'(x) + \frac{1}{2} f''(\xi) f(x)}
          \end{equation*}
          因此,
          \begin{equation*}
            \begin{split}
              \frac{\varphi(x) - \varphi(x^*)}{x - x^*}
               & = \frac{x - \frac{f(x)}{f'(x) + \frac{1}{2} f''(\xi) f(x)} - x^*}{x - x^*}      \\
               & = \frac{x - x^* - \frac{f(x)}{f'(x) + \frac{1}{2} f''(\xi) f(x)}}{x - x^*}      \\
               & = 1 - \frac{f(x) - f(x^*)}{x - x^*} \frac{1}{f'(x) - \frac{1}{2} f''(\xi) f(x)}
            \end{split}
          \end{equation*}
          所以,
          \begin{equation*}
            \varphi'(x^*)
            = \lim_{x \to x^*} \frac{\varphi(x) - \varphi(x^*)}{x - x^*}
            = 1 - f'(x^*) \frac{1}{f'(x^*)}
            = 0
          \end{equation*}
          因为 $f'(x^*) \neq 0$, 所以, Steffensen 方法至少是二阶收敛的.
  \end{enumerate}
\end{sol}

\begin{prob}[\booktitle P99 第三章 13.]
  对于 $\bm{A} =
    \begin{bmatrix}
      1  & -a \\
      -a & 1  \\
    \end{bmatrix}
  $, 写出解方程组 $\bm{A} \bm{x} = \bm{b}$ 的 J 迭代法和 SOR 迭代法的迭代矩阵 $\bm{B}_J$ 和 $\mathscr{L}_{\omega}$, 并求 $\rho(\mathscr{L}_{\omega})$ 和最优松弛因子 $\omega_b$.
\end{prob}
\begin{sol}
  % 提示: $\rho(\mathscr{L}_{\omega}) = \frac{1}{2} [2 (1 - \omega) + \omega^2 a^2 + \omega \abs{a} \sqrt{4 (1 - \omega) + \omega^2 a^2}]$, $\omega_b = \frac{1}{1 + \sqrt{1 - a^2}}$.
  \item[\emph{Jacobi 迭代法}]
  Jacobi 迭代法的迭代矩阵 $\bm{B}_J$ 为:
  \begin{equation*}
    \bm{B}_J = -\bm{D}^{-1} (\bm{L} + \bm{U})
  \end{equation*}
  其中, $\bm{D}$ 是 $\bm{A}$ 的对角线部分, $\bm{L}$ 是 $\bm{A}$ 的下三角部分, $\bm{U}$ 是 $\bm{A}$ 的上三角部分.
  \begin{equation*}
    \bm{D} = \mqty[ 1 & 0 \\ 0 & 1 ] \qc
    \bm{L} = \mqty[ 0 & 0 \\ -a & 0 ] \qc
    \bm{U} = \mqty[ 0 & -a \\ 0 & 0 ]
  \end{equation*}
  因此,
  \begin{equation*}
    \bm{B}_J =
    \mqty[0 & a \\ a & 0]
  \end{equation*}
  \item[\emph{SOR 迭代法}]
  SOR 迭代法的迭代矩阵 $\mathscr{L}_{\omega}$ 为:
  \begin{equation*}
    \mathscr{L}_{\omega} = (\bm{D} - \omega \bm{L})^{-1} \pqty{ (1 - \omega) \bm{D} + \omega \bm{U} }
  \end{equation*}
  首先计算 $\bm{D} - \omega \bm{L}$ 和 $(1 - \omega) \bm{D} + \omega \bm{U}$:
  \begin{equation*}
    \bm{D} - \omega \bm{L} =
    \begin{bmatrix}
      1        & 0 \\
      \omega a & 1
    \end{bmatrix}
    \qc
    (1 - \omega) \bm{D} + \omega \bm{U} =
    \begin{bmatrix}
      1 - \omega & - \omega a \\
      0          & 1 - \omega
    \end{bmatrix}
  \end{equation*}
  然后求逆矩阵:
  \begin{equation*}
    (\bm{D} - \omega \bm{L})^{-1} =
    \begin{bmatrix}
      1          & 0 \\
      - \omega a & 1
    \end{bmatrix}
  \end{equation*}
  最后, 迭代矩阵 $\mathscr{L}_{\omega}$ 为:
  \begin{equation*}
    \mathscr{L}_{\omega} =
    \begin{bmatrix}
      1 - \omega              & - \omega a                  \\
      - \omega a (1 - \omega) & \omega^2 a^2 + (1 - \omega)
    \end{bmatrix}
  \end{equation*}
  \item[\emph{谱半径和最优松弛因子}]
  Jacobi 迭代法的谱半径 $\rho(\bm{B}_J)$ 为:
  \begin{equation*}
    \rho(\bm{B}_J) = \abs{a}
  \end{equation*}
  SOR 迭代法的谱半径 $\rho(\mathscr{L}_{\omega})$ 为:
  \begin{equation*}
    \rho(\mathscr{L}_{\omega}) = \frac{1}{2} \bqty{2 (1 - \omega) + \omega^2 a^2 + \omega \abs{a} \sqrt{4 (1 - \omega) + \omega^2 a^2}}
  \end{equation*}
  最优松弛因子 $\omega_b$ 为:
  % \todo[inline]{$\omega_b = ???$}
  \begin{equation*}
    \omega_b = \frac{2}{1 + \sqrt{1 - a^2}}
  \end{equation*}
\end{sol}

\begin{prob}[\booktitle P98 第三章 6.]
  对于方程组 $\bm{A} \bm{x} = \bm{b}$, 试证明:
  \begin{enumerate}
    \item 若 $\bm{A}$ 严格对角占优, 则 J 法收敛 (这是定理 2.1 的一部分).
    \item 若 $\bm{A}$ ``列严格对角占优'', 即
          \begin{equation*}
            \abs{a_{jj}} > \sum_{i \neq j} \abs{a_{ij}} \qc
            j = 1, 2, \cdots, n,
          \end{equation*}
          则 J 法收敛.
  \end{enumerate}
\end{prob}
\begin{sol} ~
  % 提示:
  % (2) 若 $\bm{B}_J$ 特征值 $\lambda$, $\abs{\lambda} \geqslant 1$, 则 $\det[\bm{D} - \lambda^{-1} (\bm{L} + \bm{U})] = 0$, 而 $\bm{D} - \lambda^{-1} (\bm{L} + \bm{U})$ 亦列严格对角占优, 其转置的行列式不为 0, 矛盾.
  \begin{enumerate}
    \item Jacobi 迭代矩阵为 $\bm{B}_J = -\bm{D}^{-1}(\bm{L} + \bm{U})$, 其中 $\bm{D}$ 是 $\bm{A}$ 的对角线部分, $\bm{L}$ 是下三角部分, $\bm{U}$ 是上三角部分.
          假设 $\bm{B}_J$ 有一个特征值 $\lambda$ 且 $\abs{\lambda} \geqslant 1$, 则有 $\det(\bm{D} - \lambda^{-1}(\bm{L} + \bm{U})) = 0$.
          由于 $\bm{A}$ 严格对角占优, $\bm{D} - \lambda^{-1}(\bm{L} + \bm{U})$ 也是严格对角占优的, 因此其行列式不为零, 矛盾.
          因此, 所有特征值满足 $\abs{\lambda} < 1$, 即 $\rho(\bm{B}_J) < 1$, Jacobi 迭代法收敛.
    \item 同样假设 $\bm{B}_J$ 有一个特征值 $\lambda$ 且 $\abs{\lambda} \geqslant 1$, 则 $\det(\bm{D} - \lambda^{-1}(\bm{L} + \bm{U})) = 0$.
          由于 $\bm{A}$ 列严格对角占优, $\bm{D} - \lambda^{-1}(\bm{L} + \bm{U})$ 也是列严格对角占优的, 其转置是行严格对角占优的, 因此行列式不为零, 矛盾.
          因此, 所有特征值满足 $\abs{\lambda} < 1$, 即 $\rho(\bm{B}_J) < 1$, Jacobi 迭代法收敛.
  \end{enumerate}
\end{sol}

\begin{prob}[\booktitle P98 第三章 8.]
  分析方程组
  \begin{equation*}
    \begin{bmatrix}
      1 & a & 0 \\
      a & 1 & a \\
      0 & a & 1 \\
    \end{bmatrix}
    \mqty[\xmat*{x}{3}{1}]
    =
    \mqty[\xmat*{b}{3}{1}]
  \end{equation*}
  J 迭代法和 GS 迭代法的收敛性.
\end{prob}
\begin{sol}
  % 答案: $a \in \left(-\frac{1}{\sqrt{2}}, \frac{1}{\sqrt{2}}\right)$ 时收敛
  $\bm{A}$ 对称, 对角元素都大于零.
  顺序主子式 $\Delta_1 = 1$, $\Delta_2 = 1 - a^2$, $\Delta_3 = 1 - 2 a^2$.
  $\bm{A}$ 正定的充分必要条件是 $\Delta_2 > 0$, $\Delta_3 > 0$.
  这等价于 $a \in \pqty{-\frac{1}{\sqrt{2}}, \frac{1}{\sqrt{2}}}$.
  根据定理 2.3 和定理 2.4, GS 法收敛的充分必要条件是 $a \in \pqty{-\frac{1}{\sqrt{2}}, \frac{1}{\sqrt{2}}}$.
  \begin{equation*}
    2 \bm{D} - \bm{A} =
    \begin{bmatrix}
      1  & -a & 0  \\
      -a & 1  & -a \\
      0  & -a & 1
    \end{bmatrix}
  \end{equation*}
  其顺序主子式 $\Delta_1 = 1$, $\Delta_2 = 1 - a^2$, $\Delta_3 = 1 - 2 a^2$.
  可得 $2 \bm{D} - \bm{A}$ 正定的充分必要条件是 $a \in \pqty{-\frac{1}{\sqrt{2}}, \frac{1}{\sqrt{2}}}$.
  根据定理 2.2, J 法收敛的充分必要条件是
  \begin{equation*}
    a \in \pqty{-\frac{1}{\sqrt{2}}, \frac{1}{\sqrt{2}}} \cap \pqty{-\frac{1}{\sqrt{2}}, \frac{1}{\sqrt{2}}}
  \end{equation*}
  即 $a \in \pqty{-\frac{1}{\sqrt{2}}, \frac{1}{\sqrt{2}}}$.
\end{sol}

\begin{prob}[\booktitle P67 第二章 4.]
  设 $\bm{A} \in \mathbb{R}^{n \times n}$, $\bm{A} = [a_{ij}]$, $a_{11} \neq 0$.
  方程组 $\bm{A} \bm{x} = \bm{b}$ 经一步 Gauss 消去变换为 $\bm{A}^{(2)} \bm{X} = \bm{b}^{(2)}$, 其中
  \begin{equation*}
    \bm{A}^{(2)} =
    \begin{bmatrix}
      a_{11} & \bm{\alpha}_1^T \\
      \bm{0} & \bm{A}_2
    \end{bmatrix}
    \qc
    \bm{A}_2 =
    \begin{bmatrix}
      a_{22}^{(2)} & \cdots & a_{2n}^{(2)} \\
      \vdots       &        & \vdots       \\
      a_{n2}^{(2)} & \cdots & a_{nn}^{(2)}
    \end{bmatrix}
    .
  \end{equation*}
  试证明
  \begin{enumerate}
    \item 若 $\bm{A}$ 对称正定, 则 $\bm{A}_2$ 也对称正定.
    \item 若 $\bm{A}$ 严格对角占优, 则 $\bm{A}_2$ 也严格对角占优.
  \end{enumerate}
\end{prob}
\begin{sol} ~
  % 提示:
  % (1) 证 $\bm{A}_2$ 对称及其各阶顺序主子式大于零.
  % (2) 证 $\abs{a_{ii}^{(2)}} - \sum_{\substack{j = 2 \\ j \neq i}}^n \abs{a_{ij}^{(2)}} > 0$.
  \begin{enumerate}
    \item 若 $\bm{A}$ 对称正定, 则 $\bm{A}$ 的所有顺序主子式大于零, 且 $\bm{A}$ 可以表示为 $A =
            \begin{bmatrix}
              a_{11}        & \bm{\alpha}_1^T \\
              \bm{\alpha}_1 & \bm{A}_1
            \end{bmatrix}
          $, 其中 $\bm{A}_1$ 是 $(n - 1) \times (n - 1)$ 子矩阵.
          根据 Schur 补的性质, $\bm{A}_2 = \bm{A}_1 - \frac{\bm{\alpha_1} \bm{\alpha}_1^T}{a_{11}}$ 也是对称正定的.
          对于任意非零向量 $\bm{y} \in \mathbb{R}^{n - 1}$, 有
          \begin{equation*}
            \bm{y}^T \bm{A}_2 \bm{y} = \bm{y}^T \bm{A}_1 \bm{y} - \frac{(\bm{\alpha}_1^T \bm{y})^2}{a_{11}}.
          \end{equation*}
          由于 $\bm{A}$ 是对称正定的, 对于任意非零向量 $\mqty[x_1 \\ \bm{y}]$, 有
          \begin{equation*}
            x_1^2 a_{11} + 2 x_1 \bm{y}^T \bm{\alpha}_1 + \bm{y}^T \bm{A}_1 \bm{y} > 0.
          \end{equation*}
          选择 $x_1 = -\frac{\bm{y}^T \bm{\alpha}_1}{a_{11}}$, 则
          \begin{equation*}
            \pqty{-\frac{\bm{y}^T \bm{\alpha}_1}{a_{11}}}^2 a_{11} + 2 \pqty{-\frac{\bm{y}^T \bm{\alpha}_1}{a_{11}}} \bm{y}^T \bm{\alpha}_1 + \bm{y}^T \bm{A}_1 \bm{y} > 0
          \end{equation*}
          化简得
          \begin{equation*}
            \bm{y}^T \bm{A}_2 \bm{y} > 0
          \end{equation*}
          因此 $\bm{A}_2$ 是对称正定的.
    \item 根据高斯消元的定义, $a_{ij}^{(2)} = a_{ij} - \frac{a_{i1}}{a_{11}} a_{1j}$, 其中 $i, j$ 从 $2$ 到 $n$.
          我们通过以下不等式推导来证明:
          \begin{equation*}
            \sum_{\substack{j=2 \\ j \neq i}}^n \abs{a_{ij}^{(2)}} \leqslant \sum_{\substack{j=2 \\ j \neq i}}^n \abs{a_{ij}} + \frac{\abs{a_{i1}}}{\abs{a_{11}}} \sum_{\substack{j=2 \\ j \neq i}}^n \abs{a_{1j}}
          \end{equation*}
          利用 $\bm{A}$ 的严格对角占优性, 有:
          \begin{equation*}
            \sum_{j=2}^n \abs{a_{1j}} < \abs{a_{11}}
          \end{equation*}
          因此:
          \begin{equation*}
            \sum_{\substack{j=2 \\ j \neq i}}^n \abs{a_{1j}} < \abs{a_{11}} - \abs{a_{1i}}
          \end{equation*}
          代入上式:
          \begin{equation*}
            \sum_{\substack{j=2 \\ j \neq i}}^n \abs{a_{ij}^{(2)}} < \abs{a_{ii}} - \abs{a_{i1}} + \frac{\abs{a_{i1}}}{\abs{a_{11}}} \pqty{\abs{a_{11}} - \abs{a_{1i}}}
          \end{equation*}
          进一步化简:
          \begin{equation*}
            \sum_{\substack{j=2 \\ j \neq i}}^n \abs{a_{ij}^{(2)}} < \abs{a_{ii}} - \frac{\abs{a_{i1}}}{\abs{a_{11}}} \abs{a_{1i}}
          \end{equation*}
          而:
          \begin{equation*}
            \abs{a_{ii} - \frac{a_{i1}}{a_{11}} a_{1i}} \geqslant \abs{a_{ii}} - \frac{\abs{a_{i1}}}{\abs{a_{11}}} \abs{a_{1i}}
          \end{equation*}
          因此:
          \begin{equation*}
            \sum_{\substack{j=2 \\ j \neq i}}^n \abs{a_{ij}^{(2)}} < \abs{a_{ii} - \frac{a_{i1}}{a_{11}} a_{1i}} = \abs{a_{ii}^{(2)}}
          \end{equation*}
          这证明了 $\bm{A}_2$ 也是严格对角占优的.
          % \begin{equation*}
          %   \begin{split}
          %     \sum_{\substack{j = 2                                                                                   \\ j \neq i}}^n \abs{a_{ij}^{(2)}}
          %       & = \sum_{\substack{j = 2                                                                              \\ j \neq i}}^n \abs{a_{ij} - \frac{1}{a_{11}} a_{i1} a_{1j}} \\
          %       & \leqslant \sum_{\substack{j = 2                                                                      \\ j \neq i}}^n \abs{a_{ij}} + \frac{\abs{a_{i1}}}{\abs{a_{11}}} \sum_{\substack{j = 2 \\ j \neq i}}^n \abs{a_{1j}} \\
          %       & < \abs{a_{ii}} - \abs{a_{i1}} + \frac{\abs{a_{i1}}}{\abs{a_{11}}} \sum_{\substack{j = 2              \\ j \neq i}}^n \abs{a_{1j}} \\
          %       & < \abs{a_{ii}} - \abs{a_{i1}} + \frac{\abs{a_{i1}}}{\abs{a_{11}}} \pqty{\abs{a_{11}} - \abs{a_{1i}}} \\
          %       & = \abs{a_{ii}} - \frac{\abs{a_{i1}}}{\abs{a_{11}}} \abs{a_{1i}}                                      \\
          %       & \leqslant \abs{a_{ii} - \frac{a_{i1}}{a_{11}} a_{1i}}
          %     = \abs{a_{ii}^{(2)}}
          %   \end{split}
          % \end{equation*}
  \end{enumerate}
\end{sol}

\begin{prob}[\booktitle P132 第四章 5.]
  设 $f \in C^1(\mathbb{R})$, 满足 $f(x^*) = 0$, $0 < m \leqslant f'(x) \leqslant M$.
  试证明迭代法
  \begin{equation*}
    x_{k + 1} = x_k - \lambda f(x_k)
  \end{equation*}
  产生的序列 $\{x_k\}$ 对任意的 $x_0 \in \mathbb{R}$ 及 $\lambda \in \pqty{0, \frac{2}{M}}$ 均收敛到 $x^*$.
\end{prob}
\begin{sol}
  % 提示: $\varphi(x) = 1 - \lambda f(x)$, $L = \max\{\abs{1 - \lambda M}, \abs{1 - \lambda M}\}$, $\abs{\varphi'(x) \leqslant L}$, $\abs{x_k - x^*} \leqslant L^k \abs{x_0 - x^*}$.
  首先, 我们考虑迭代公式 $x_{k+1} = x_k - \lambda f(x_k)$, 将其写成不动点迭代形式 $x_{k+1} = \varphi(x_k)$, 其中 $\varphi(x) = x - \lambda f(x)$.
  接下来, 计算 $\varphi(x)$ 的导数:
  \begin{equation*}
    \varphi'(x) = 1 - \lambda f'(x)
  \end{equation*}
  由于 $f'(x)$ 在区间 \([m, M]\) 上, 因此 $\varphi'(x)$ 的取值范围为 $[1 - \lambda M, 1 - \lambda m]$.
  为了确保迭代法的收敛性, 我们需要 $\abs{\varphi'(x)} < 1$.
  我们考虑 $\varphi'(x)$ 的绝对值的最大值:
  \begin{equation*}
    L = \max \{ \abs{1 - \lambda m}, \abs{1 - \lambda M} \}
  \end{equation*}
  我们需要 $L < 1$, 即:
  \begin{equation*}
    \abs{1 - \lambda m} < 1 \qq{和} \abs{1 - \lambda M} < 1
  \end{equation*}
  对于 $\abs{1 - \lambda m} < 1$, 有:
  \begin{equation*}
    -1 < 1 - \lambda m < 1 \implies \lambda m < 2 \implies \lambda < \frac{2}{m}
  \end{equation*}
  对于 $\abs{1 - \lambda M} < 1$, 有:
  \begin{equation*}
    -1 < 1 - \lambda M < 1 \implies \lambda M < 2 \implies \lambda < \frac{2}{M}
  \end{equation*}
  由于 $m \leqslant M$, 因此 $\frac{2}{m} \geqslant \frac{2}{M}$, 所以更严格的条件是 $\lambda < \frac{2}{M}$.
  因此, 当 $\lambda \in \left(0, \frac{2}{M}\right)$ 时,  $\abs{\varphi'(x)} < 1$, 即迭代法是压缩映射.
  根据 Banach 不动点定理, 序列 $\{x_k\}$ 将收敛到唯一的不动点 $x^*$, 即对于任意 $x_0 \in \mathbb{R}$ 和 $\lambda \in \left(0, \frac{2}{M}\right)$, 序列 $\{x_k\}$ 均收敛到 $x^*$.
  综上所述, 迭代法 $x_{k+1} = x_k - \lambda f(x_k)$ 在给定条件下收敛到 $x^*$.
\end{sol}

\end{document}
